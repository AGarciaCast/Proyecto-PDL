
\section{Diseño del Analizador Léxico}

\subsection{Tokens}
\indent

$<$PuntoComa, - $>$

$<$CorcheteAbrir, - $>$ 

$<$CorcheteCerrar, - $>$

$<$ID, posTS$>$ (Identificador)

$<$ENT, valor$>$ (Dato de tipo entero)

$<$CAD, lex$>$ (Dato de tipo cadena)

$<$ParentesisCerrar, - $>$

$<$ParentesisAbrir, - $>$

$<$SUMA, - $>$ (Operador suma)

$<$MENOR, - $>$ (Operador lógico menor)

$<$NOT, - $>$ (Operador lógico de negación)

$<$ASIG, - $>$ (Operador de asignación)

$<$ASIGOR, - $>$ (Asignación con o lógico)

$<$DEC, - $>$ (“var”)

$<$TipoVarENT, - $>$ (“int”)

$<$TipoVarLOG, - $>$ (“boolean”)

$<$TipoVarCAD, - $>$ (“string”)

$<$Print, - $>$

$<$Input, - $>$

$<$Coma, - $>$

$<$Return, -$>$

$<$DECFunc, - $>$ (“function”)

$<$IF, - $>$

$<$ELSE, - $>$
\subsection{Gramática}

\begin{lstlisting}[style = Gramatica]
G(N, T, S, P)

S = A
N = { A, B, C, D, E, F, G, H }
T = { del, ;, {, }, (, ), +, <, !, =, ,, l, d, ', /, _, *, c }   

P:
A -> delA | ; | { | } | ( | ) | + | < | ! | = | ,
A -> |B | lC | dD | 'E | /F
B -> =
C -> lC | dD | _C | λ
D -> dD | λ
E -> cE | *E | /E | '
F -> *G
G -> cG | /G | *H
H -> /A | cG | *H
Donde, c = T - {*, /}
\end{lstlisting}

\subsection{Autómata Finito Determinista}
\begin{tikzpicture}[->,thick, node distance= 2.5cm,auto]
\tikzstyle{nodoChiquito} = [state, accepting, node distance = 1.5cm];
\node [state, initial](0){$0$};

\node [state](1)[right of = 0, xshift=2cm]{$1$};
\node [state, accepting](8)[right of = 1]{$8$};

\node [state](5)[above of = 1,yshift=1.5cm]{$5$};
\node [state](6)[right of = 5]{$6$};
\node [state](7)[below of = 6]{$7$};

\node [state](2)[below of = 1]{$2$};
\node [state, accepting](9)[right of = 2]{$9$};

\node [state](3)[below of = 2,]{$3$};
\node [state, accepting](10)[right of = 3]{$10$};

\node [state](4)[below of = 3]{$4$};
\node [state, accepting](11)[right of = 4]{$11$};

\node [nodoChiquito](12)[left of = 0,yshift=7.5cm,xshift=-3cm]{$12$};
\node [nodoChiquito](13)[below of = 12]{$13$};
\node [nodoChiquito](14)[below of = 13]{$14$};
\node [nodoChiquito](15)[below of = 14]{$15$};
\node [nodoChiquito](16)[below of = 15]{$16$};
\node [nodoChiquito](17)[below of = 16,yshift=-1.5cm]{$17$};
\node [nodoChiquito](18)[below of = 17]{$18$};
\node [nodoChiquito](19)[below of = 18]{$19$};
\node [nodoChiquito](20)[below of = 19]{$20$};
\node [nodoChiquito](21)[below of = 20]{$21$};

\path 
	(0) edge[loop above] node{del} (0)
        edge[bend right=73, right] node{; {\color{blue} G5}} (12)
        edge[bend right=55, right] node{\{ {\color{blue} G6}} (13)
        edge[bend right=30, right] node{\} {\color{blue} G7}} (14)
        edge[bend right=20, above] node{( {\color{blue} G8}} (15)
        edge[bend right=10, above] node{) {\color{blue} G9}} (16)
        edge[bend left=20, above] node{+ {\color{blue} G10}} (17)
        edge[bend left=20, left] node{$<$ {\color{blue} G11}} (18)
        edge[bend left=30, left] node{! {\color{blue} G12}} (19)
        edge[bend left=40, left] node{= {\color{blue} G13}} (20)
        edge[bend left=50, left] node{, {\color{blue} G14}} (21)
            
	    
	(0) edge [bend right=12] node{$\vert$} (1)
	(1) edge node{= {\color{blue} G1}} (8)
	
	(0) edge[bend right = 15] node{l {\color{blue} C}} (2)
	(2) edge [loop above] node{e $\vert$ d $\vert$ - {\color{blue} C} } (2)
	    edge node{o.c {\color{blue} G2}} (9)
	
	(0) edge[bend right = 20] node{d {\color{blue} A}} (3)
	(3) edge[loop above] node{d {\color{blue} B}} (3)
	    edge node{o.c {\color{blue} G3}} (10)
	    
	(0) edge[bend right=12,left] node{' {\color{blue} D}} (4)
	(4) edge[loop above] node{c $\vert$ * $\vert$ / {\color{blue} E}} (4)
	    edge node{' {\color{blue} G4}} (11)
	
	(0) edge node{/} (5)
	(5) edge node{*} (6)
	(6) edge [bend left] node{*} (7)
	    edge [loop right] node{c $\vert$ /} (6)
	(7) edge [bend left] node {c} (6)
	    edge [loop right] node{*} (7)
	    edge[bend right=12, above] node {/} (0)
;
\end{tikzpicture}
\newpage
\subsection{Acciones Semánticas}

\begin{lstlisting}[style=AccionesSemanticas]
Lee $\forall$ transicion menos o.c

C:  CONCAT()

$G_1$:  GEN_TOKEN(ASIGOR, -)

$G_2$: if(lex $\in$ palRes) GEN_TOKEN(palRes, -)
else if(FlagDeclUso = Decl)
	if (estaEnTSActual(lex)) 
		Error("Variable ya declarada")
	else 
		p = INSERTAR_TS(lex)
		GEN_TOKEN(ID, p)
else 
	p = BUSCA_TS(lex)
	if (p = null) p = INSERTAR_TS(lex)
	GEN_TOKEN(ID, p)

A: num = valor(d)

B: num = num * 10 + valor(d)

D: cont = 0

E: cont = cont + 1
   CONCAT()

$G_3$: if(num $>=$ $2^{15}$) Error("Numero se sale del rango")
    else GEN_TOKEN(ENT, num)

$G_4:$ if(cont $>$ 64) Error("Exceso de caracteres en la cadena")
     else GEN_TOKEN(CAD, lex)

$G_5:$ GEN_TOKEN(PuntoComa, -)

$G_6:$ GEN_TOKEN(CorcheteAbrir, -)

$G_7:$ GEN_TOKEN(CorcheteCerrar, -)

$G_8:$ GEN_TOKEN(ParentesisAbrir, -)

$G_9:$ GEN_TOKEN(ParentesisCerrar, -)

$G_{10}:$ GEN_TOKEN(SUMA, -)

$G_{11}:$ GEN_TOKEN(MENOR, -)

$G_{12}:$ GEN_TOKEN(NOT, -)

$G_{13}:$ GEN_TOKEN(ASIG, -)

$G_{14}:$ GEN_TOKEN(Coma, -)	
\end{lstlisting}
Donde, palRes $=$ $\{$var, int, boolean, string, print, input, function, return, if, else$\}$
\newline

\subsection{Errores}
\noindent Error 1: "Transición no prevista.";\\ 
\noindent Error 2: "Numero fuera de rango.";\\ 
\noindent Error 3: "Exceso de caracteres en la cadena.";\\ 
\noindent Error 4: "Variable ya declarada.";\\ 

\subsection{Matriz de Transiciones}
\hspace*{-50pt}\setlength{\tabcolsep}{0.7\tabcolsep} \begin{tabular}{|c|c|c|c|c|c|c|c|c|c|}
\hline
    \textbf{MT\_AFD} & \textbf{\textbar}  & \textbf{letra} & \textbf{digito} & \textbf{'}     & \textbf{/}     & \textbf{\_}    & \textbf{carácter} & \textbf{*}     & \textbf{delimitador} \\
\hline
 $\rightarrow$ 0     & 1 lee & 2 C   & 3A    & 4 D & 5 lee & -1 error & -1 error & -1 error & 0 lee \\
\hline
    1     & -1 error & -1 error & -1 error & -1 error & -1 error & -1 error & -1 error & -1 error & -1 error \\
\hline
    2     & 9 G2  & 2 C   & 2 C   & 9 G2  & 9 G2  & 2 C   & 9 G2  & 9 G2  & 9 G2 \\
\hline
    3     & 10 G3 & 10 G3 & 3 B   & 10 G3 & 10 G3 & 10 G3 & 10 G3 & 10 G3 & 10 G3 \\
\hline
    4     & 4 E   & 4 E   & 4 E   & 11 G4 & 4 E   & 4 E   & 4 E   & 4 E   & 4 E  \\
\hline
    5     & -1 error & -1 error & -1 error & -1 error & -1 error & -1 error & -1 error & 6 lee & -1 error \\
\hline
    6     & 6 lee & 6 lee & 6 lee & 6 lee & 6 lee & 6 lee & 6 lee & 7 lee & 6 lee \\
\hline
    7     & 6 lee & 6 lee & 6 lee & 6 lee & 0 lee & 6 lee & 6 lee & 7 lee & 6 lee \\
\hline
    \end{tabular}\hspace{-50pt}\\\\

\hspace*{-70pt} \begin{tabular}{|c|c|c|c|c|c|c|c|c|c|c|}
\hline
    \textbf{MT\_AFD}    & \textbf{;}  & \textbf{\{}     & \textbf{\}}     & \textbf{(}     & \textbf{)}     & \textbf{+}     & \textbf{\textless}     & \textbf{!}     & \textbf{=}     & \textbf{,} \\
\hline
$\rightarrow$ 0 & 12 G5 & 13 G6 & 14 G7 & 15 G8 & 16 G9 & 17 G10 & 18 G11 & 19 G12 & 20 G13 & 21 G14 \\
\hline
    1  & -1 error & -1 error & -1 error & -1 error & -1 error & -1 error & -1 error & -1 error & 8 G1  & -1 error \\
\hline
    2 & 9 G2 & 9 G2  & 9 G2  & 9 G2  & 9 G2  & 9 G2  & 9 G2  & 9 G2  & 9 G2  & 9 G2 \\
\hline
    3 & 10 G3 & 10 G3 & 10 G3 & 10 G3 & 10 G3 & 10 G3 & 10 G3 & 10 G3 & 10 G3 & 10 G3 \\
\hline
    4 & 4 E & 4 E   & 4 E   & 4 E   & 4 E   & 4 E   & 4 E   & 4 E   & 4 E   & 4 E \\
\hline
    5 & -1 error & -1 error & -1 error & -1 error & -1 error & -1 error & -1 error & -1 error & -1 error & -1 error \\
\hline
    6 & 6 lee & 6 lee & 6 lee & 6 lee & 6 lee & 6 lee & 6 lee & 6 lee & 6 lee & 6 lee \\
\hline
    7 & 6 lee & 6 lee & 6 lee & 6 lee & 6 lee & 6 lee & 6 lee & 6 lee & 6 lee & 6 lee \\
\hline
    \end{tabular}\hspace{-70pt}

    
\section{Tabla de Símbolos: Estructura e implementación}
Contiene la información de los identificadores, de los cuales se guardan los campos: lexema, tipo y desplazamiento.
Para las funciones, además, se guardará el número de parámetros, su tipo, la forma de paso de parámetros, el tipo del valor de retorno y etiqueta de función (formada por nombre y su posición en tabla de simbolos).

La tabla de símbolos estará formada por dos matrices de tamaño dinámico; la primera contendrán los identificadores de ámbito global y la segunda del local. Así pues, esta segunda se creará al encontrar la declaración de una función y se borrará al acabar de ser declarada. También se utiliza un flag de declaración o uso (FlagDeclUso), un flag para saber cual es la tabla actual y dos más para el valor del desplazamiento en cada una de las tablas.
