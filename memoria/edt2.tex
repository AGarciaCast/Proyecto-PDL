\documentclass[a4paper, 12pt]{article}
\usepackage[utf8]{inputenc}
\usepackage{listings}
\lstdefinestyle{EstadosAutomataST}{
  basicstyle=\mdseries\footnotesize,
  xleftmargin=0em,
  mathescape=true,
  tabsize=4,
  literate={->}{$\rightarrow$}{2}
           {λ}{$\lambda$}{1}

}

\lstset{
  basicstyle=\mdseries,
  xleftmargin=0cm,
  mathescape=true
}

\begin{document}

\begin{lstlisting}[style=EstadosAutomataST]


0.- P' -> {TSG = creaTS(); DesplG = 0; TS_actual = TSG;
    P.func = false} P {liberaTS(TSG)}
1.- P -> D { P$_1$ .func = P.func} P$_{1}$
2.- P -> F { P$_1$ .func = P.func} P$_{1}$
3.- P -> {S.func = false} S {P$_1$ = P.func} P$_1$
4.- D -> var {zona_decl = true} T id ;
    {InsertarTipoTS(id.posi, T.tipo)
        if(TS_actual==TSG) then
            InsertarDespl(id.posi, desplG)
            desplG = desplG + T.tamaño
        else
            InsertarDespl(id.posi, desplL)
            desplL = desplL + T.tamaño
        zona_decl=false
    }
5.- T -> int {T.tipo=entero; T.tamaño = 1}
6.- T -> string {T.tipo=cadena; T.tamaño = 64}
7.- T -> boolean {T.tipo=logico; T.tamaño = 1}
8.- F -> function {zona_decl =true} T$_1$ id({TSL = creaTS();
    desplL=0; TS_actual = TSL} A){InsertaTipoTSG}(id.posi,
    A.tipo -> T.tipo)); zona_decl = false}
    { { C.func = true} C {if (C.tipoRet != T.tipo)
    then error(1); TS_actual = TSG; LiberarTS(TSL) }}
9.- T$_1$ -> λ {T.tipo = tipo_vacio}
10.- T$_1$ -> T{T$_1$.tipo = T.tipo }
11.- A -> T id {InsertarTipoTS(id.posi, T.tipo) ;
     InsertarDesplTS(id.posi, desplL);
     desplL = desplL + T.tamaño} K
     {A.tipo = if(K.tipo == tipo_vacio)
                            then T.tipo}
                         else
                             T.tipo x K.tipo
     }
12.- A -> λ {A.tipo = tipo_vacio}
13.- K -> λ {K.tipo = tipo_vacio}
14.- K -> , T id {InsertarTipoTS(id.posi, T.tipo);
     InsertarDesplTS(id.posi, desplL);
     desplL = desplL + T.tamaño} K$_1$
     {K.tipo = if(K$_1$.tipo == tipo_vacio) then
                    T.tipo
                else
                    T.tipo x K.tipo
      }
15.- C -> D {C$_1$.func = C.func}
          C$_1$ {C.tipoRet = C$_1$.tipoRet}
16.- C -> {S.func = C.func} S {C$_1$.func = C.func}C$_1$
        {C.tipoRet =
        if(S.tipoRet == C$_1$.tipoRet) then
            S.tipoRet
        else if(S.tipoRet == tipo_vacio) then
            C$_1$.tipoRet
        else if(C$_1$.tipoRet == tipo_vacio) then
            S.tipoRet
        else
            error(2)
     }
17.- C -> λ{C.tipoRet = tipo_vacio}
18.- S-> id L E ; {S.tipo =
     if(BuscaTipoTS(id.posi) == E.tipo)
     AND (E.tipo != tipo_error))then
        tipo_ok
     else
        error(3)}
19.- S-> id (M); {S.tipo =
     if(BuscaTipoTS(id.posi) == M.tipo -> t)
        then tipo_ok
     else
        error(4)}
20.- S -> print (E); {S.tipo =
     if(E.tipo == entero || E.tipo == cadena)
        then tipo_ok
     else
        error(5)
        }
21.- S -> input(id); {S.tipo =
     if(BuscaTipoTS(id.posi) == entero
        || BuscaTipoTS(id.posi).tipo == cadena)) then tipo_ok
     else
        error(6)
     }
22.- S -> if(E) {S$_1$.func = S.func} S$_1$ {S.tipo =
     if(E.tipo == logico) then S$_1$.tipo
     else
        error(7)
     }
23.- S -> return X; {S.tipo =
     if(S.func) then
        if(X.tipo != tipo.error) then tipo_ok
        else
            error(8)
     else
        error(9)
     S.tipoRet = X.tipoRet
     }
24.- L -> |= {}
25.- L-> = {}
26.- M-> EQ {M.tipo =
     if(E.tipo != tipo_error
     AND Q.tipo != tipo_error)
        then if(Q.tipo == tipo_vacio)
                then E.tipo
             else
                E.tipo x Q.tipo
     else
        error(10)
    }
27.- M -> λ {M.tipo = tipo_vacio}
28.- Q -> λ {Q.tipo = tipo_vacio}
29.- Q -> ,EQ$_1$ {Q.tipo=
                   if(Aux[tope-1].tipo != tipo_error
                   AND Aux[tope].tipo != tipo_error)
                     then if(Aux[tope].tipo == tipo_vacio)
                            then NuevaPila(Aux[tope-1].tipo)
                          else Aux[tope].tipo.push(Aux[tope-1].tipo)
                   else
                     error(11)
                  }
30.- S$_1$ -> { {S$_2$.func = S$_1$.func} S$_2$}
     {G.func=S$_1$.func} G {S$_1$.tipo =
                            if(S$_2$.tipo != tipo_error)
                                if(G.tipo != tipo_error)
                                    then S$_2$.tipo
                                else error(13)
                            else error(12)
     }
31.- S1 -> { {S$_2$.func=s$_1$.func} S {Aux[ntope].tipo=Aux[tope].tipo}
32.- G -> else{ {S$_2$.func=G.func} S$_2$}{G.tipo=S$_2$.tipo}
33.- G -> λ {G.tipo = tipo_vacio}
34.- X -> E {X.tipo = E.tipo}
35.- X -> λ {X.tipo = tipo_vacio}
36.- E -> E$_1$ < U {E.tipo = if(E$_1$.tipo = U.tipo = entero)
                                then logico
                              else
                                error(14)
                    }
37.- E -> U {E.tipo = U.tipo}
38.- U -> U$_1$ + R {U.tipo = if(U$_1$.tipo = R.tipo = entero)
                                then entero
                                else
                                  error(15)
                    }
39.- U-> R {U.tipo = R.tipo}
40.- R -> !V {R.tipo = if(V.tipo = logico) then logico
                       else error(16)
             }
41.- R -> V {R.tipo = V.tipo}
42.- V -> (E) {V.tipo = AE.tipo}
43.- V -> id {V.tipo = BuscaTS(id.posi)}
44.- V -> id(M) {S.tipo = if(BuscatipoTS(id.posi) == M.tipo -> t)
                            then t
                          else
                            error(17)
                }
45.- V -> ent {V.tipo = entero}
46.- V -> cadena {V.tipo = cadena}
47.- S$_2$ -> {S.func = S$_2$.func} S {S'$_2$.func = S$_2$.func} S'$_2$ {
              S$_2$.tipo =
                  if(S.tipo != tipo_error) then S$_2$.tipo
                  else
                    error(18)
              }
48.- S$_2$ -> {S.func = S$_2$.func} S {S$_2$.tipo = S.tipo}
49.- P -> λ {}

\end{lstlisting}

\end{document}
