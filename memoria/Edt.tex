\documentclass[a4paper, 12pt]{article}
\usepackage[utf8]{inputenc}
\usepackage{listings}
\lstdefinestyle{EstadosAutomataST}{
  basicstyle=\mdseries\footnotesize,
  xleftmargin=0em,
  mathescape=true,
  tabsize=4,
  literate={->}{$\rightarrow$}{2}
           {λ}{$\lambda$}{1}

}

\lstset{
  basicstyle=\mdseries,
  xleftmargin=0cm,
  mathescape=true
}

\begin{document}

\begin{lstlisting}[style=EstadosAutomataST]


0.- P' -> {TSG = creaTS(); DesplG = 0; TS_actual = TSG;
    P.func = false}
    P {liberaTS(TSG)}
1.- P -> D { P$_1$ .func = P.func} P$_{1}$
2.- P -> F { P$_1$ .func = P.func} P$_{1}$
3.- P -> {S.func = false} S {P$_1$ = P.func} P$_1$
4.- D -> var {zona_decl = true} T id ;
    {InsertarTipoTS(Aux[tope-1].posi, desplG)
        if(TS_actual==TSG) then
            InsertarDespl(Aux[tope-1].posi, desplG)
            desplG = desplG + Aux[tope-2].tamaño
        else
            InsertarDespl(Aux[tope-1].posi, desplL)
            desplL = desplL + Aux[tope-2].tamaño
        zona_decl=false
    }
5.- T -> int {Aux[ntope].tipo=entero; Aux[ntope].tamaño = 1}
6.- T -> string {Aux[ntope].tipo=cadena; Aux[ntope].tamaño = 64}
7.- T -> boolean {Aux[ntope].tipo=logico; Aux[ntope].tamaño = 1}
8.- F -> function {zona_decl =true} T$_1$ id({TSL = creaTS();
    desplL=0; TS_actual = TSL} A){InsertaTipoTSG}(Aux[tope-1].posi,
    ParFunc(Aux[tope-4].tipo, Aux[tope-7].tipo)); zona_decl = false}
    { { C.func = true} C {if (Aux[tope-1].tipoRet != Aux[tope-7].tipo)
    then error(1); TS_actual = TSG; LiberarTS(TSL) }}
9.- T$_1$ -> λ {Aux[ntope].tipo = tipo_vacio}
10.- T$_1$ -> T{Aux[ntope].tipo = Aux[tope].tipo }
11.- A -> T id {InsertarTipoTS(Aux[tope-1].posi, Aux[tope-2].tipo) ;
     InsertarDesplTS(Aux[tope-1].posi, desplL);
     desplL = desplL + Aux[tope-2].tamaño} K
     {Aux[ntope].tipo = if(Aux[tope].tipo == tipo_vacio)
                            then Aux[tope-2].tipo}
                         else
                             Aux[tope].tipo.push(Aux[tope-2].tipo)
     }
12.- A -> λ {Aux[ntope].tipo = tipo_vacio}
13.- K -> λ {Aux[ntope].tipo = tipo_vacio}
14.- K -> , T id {InsertarTipoTS(Aux[tope-1].posi, Aux[tope-2].tipo);
     InsertarDesplTS(Aux[tope-1].posi, desplL);
     desplL = desplL + Aux[tope-2].tamaño} K$_1$
     {K.tipo = if(Aux[tope].tipo == tipo_vacio) then
                    NuevaPila(Aux[tope-2].tipo)
                else
                    Aux[tope].tipo.push(Aux[tope-2].tipo)
      }
15.- C -> D {C$_1$.func = C.func} 
          C$_1$ {Aux[ntope].tipoRet = Aux[tope].tipoRet}
16.- C -> {S.func = C.func} S {C$_1$.func = C.func}C$_1$
        {Aux[ntope].tipoRet=
        if(Aux[tope-1].tipoRet == Aux[tope].tipoRet) then
            Aux[tope-1].tipoRet
        else if(Aux[tope-1].tipoRet == tipo_vacio) then
            Aux[tope-1].tipoRet
        else if(Aux[tope].tipoRet == tipo_vacio) then
            Aux[tope-1].tipoRet
        else
            error(2)
     }
17.- C -> λ{Aux[ntope].tipoRet = tipo_vacio}
18.- S-> id L E ; {Aux[ntope].tipo =
     if(BuscaTipoTS(Aux[tope-3].posi)==(Aux[tope-1].tipo)
     AND (Aux[tope-1].tipo != tipo_error))then
        tipo_ok
     else
        error(3)}
19.- S-> id (M); {Aux[ntope].tipo =
     if(BuscaTipoTS(Aux[tope-4].posi)== ParFunc(Aux[tope-2].tipo, t)
        then tipo_ok
     else
        error(4)}
20.- S -> print (E) ; {Aux[ntope].tipo =
     if(Aux[tope-2].tipo == entero || Aux[tope-2].tipo == cadena)
        then tipo_ok
     else
        error(5)
        }
21.- S -> input(id); {Aux[ntope].tipo =
     if(BuscaTipoTS(Aux[tope-2].posi == entero
        || Aux[tope-2].tipo == cadena)) then tipo_ok
     else
        error(6)
     }
22.- S -> if(E) {S$_1$.func = S.func} S$_1$ {Aux[ntope].tipo =
     if(Aux[tope-2].tipo == logico) then Aux[tope].tipo
     else
        error(7)
     }
23.- S -> return X;{Aux[ntope].tipo =
     if(Aux[ntope].func) then
        if(Aux[tope-1].tipo != tipo.error) then tipo_ok
        else
            error(8)
     else
        error(9)
     }
24.- L -> |= {}
25.- L-> = {}
26.- M-> EQ {Aux[ntope].tipo =
     if(Aux[tope-1].tipo != tipo_error
     AND Aux[tope].tipo != tipo_error)
        then if(Aux[tope].tipo == tipo_vacio)
                then Aux[tope-1].tipo
             else
                Aux[tope].tipo.push(Aux[tope-1].tipo)
     else
        error(10)
    }
27.- M -> λ {Aux[ntope].tipo = tipo_vacio}
28.- Q -> λ {Aux[ntope].tipo = tipo_vacio}
29.- Q -> ,EQ$-1${if(Aux[tope-1].tipo != tipo_error
                  AND Aux[tope].tipo != tipo_error)
                    then if(Aux[tope].tipo == tipo_vacio)
                            then NuevaPila(Aux[tope-1].tipo)
                         else Aux[tope].tipo.push(Aux[tope-1].tipo)
                  else
                    error(11)
                  }
30.- S$-1$ -> { {S$_2$.func = S$_1$.func} S$_2$}
     {G.func=S$_1$.func} G {Aux[ntope].tipo =
                            if(Aux[tope-2].tipo != tipo_error)
                                if(Aux[tope].tipo != tipo_error)
                                    then Aux[tope-2].tipo
                                else error(13)
                            else error(12)
     }
31.- S1 -> { {S$_2$.func=s$_1$.func} S {Aux[ntope].tipo=Aux[tope].tipo}
32.- G -> else{ {S$_2$.func=G.func} S$_2$}{Aux[ntope].tipo=Aux[tope-1].tipo}
33.- G -> λ {Aux[ntope].tipo = tipo_vacio}
34.- X -> E {Aux[ntope].tipo = Aux[tope].tipo}
35.- X -> λ {Aux[ntope].tipo = tipo_vacio}
36.- E -> E$_1$ < U {Aux[ntope].tipo =
                     if(Aux[tope-2].tipo = Aux[tope].tipo = entero)
                        then logico
                     else
                        error(14)
                    }
37.- E -> U {Aux[ntope].tipo = Aux[tope].tipo}
38.- U -> u$_1$ + R {Aux[ntope].tipo =
                    if(Aux[tope-2].tipo = Aux[tope].tipo = entero)
                        then entero
                    else
                        error(15)
                    }
39.- U-> R {Aux[ntope].tipo = Aux[tope].tipo}
40.- R _> !V {Aux[ntope].tipo =
             if(Aux[tope].tipo = logico) then logico
             else error(16)
             }
41.- R -> V {Aux[ntope].tipo = Aux[tope].tipo}
42.- V -> (E) {Aux[ntope].tipo = Aux[tope-1].tipo}
43.- V -> id {Aux[ntope].tipo = BuscaTipoTS(Aux[tope].posi)}
44.- V -> id(M) {Aux[ntope].tipo =
                 if(BuscaTipoTS(Aux[tope].posi) == ParFunc(Aux[tope-1].tipo, t))
                    then t
                 else
                    error(17)
                }
45.- V -> ent{Aux[ntope].tipo = entero}
46.- V -> cadena{Aux[ntope].tipo = cadena}
47.- S$_2$ -> {S.func = S$_2$.func} S {S'$_2$.func = S$_2$.func} S'$_2$ {
              Aux[ntope] = if(Aux[tope-1].tipo != tipo_error) then Aux[tope].tipo
                  else
                    error(18)
              }
48.- S$_2$ -> {S.func = S$_2$.func} S {Aux[ntope].tipo = Aux[tope].tipo}
49.- P -> λ {}

\end{lstlisting}

\end{document}
